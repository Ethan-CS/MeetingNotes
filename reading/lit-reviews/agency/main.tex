\documentclass[11pt]{amsart}
\usepackage{geometry}                % See geometry.pdf to learn the layout options
	\geometry{a4paper}  

\usepackage{enumitem}
\usepackage{graphicx}
\usepackage{subcaption}
\usepackage{amsmath}
\usepackage{amssymb}
\usepackage{epstopdf}
\usepackage{setspace}
\usepackage{listings}

\lstdefinestyle{myListingStyle} 
    {
        basicstyle = \small\ttfamily,
        breaklines = true,
    }

\DeclareRobustCommand{\subtitle}[1]{\\#1}

\graphicspath{ {./assets/} }

\DeclareGraphicsRule{.tif}{png}{.png}{`convert #1 `dirname #1`/`basename #1 .tif`.png}
%\setkomafont{title}{\normalfont\scshape}
%\addtokomafont{disposition}{\rmfamily}

%\newcommand{\subtitle}{A review of previous work and avenues for future research}
    
\title{Agency in Graphical Models of Contagion}
%\subtitle{A discussion of possible means to introduce agency into graph problems used to model contagion}
\author{Ethan Kelly}
%\author{Supervisor: Jessica Enright}

\newcommand{\Addresses}{{
  \bigskip
  \footnotesize
  \textsc{Ethan Kelly} \par\nopagebreak
  	\hspace{\parindent}\textit{Email:~}\texttt{E.Kelly.1@research.gla.ac.uk}\par\nopagebreak
	\vspace{1mm}
  \textsc{Supervisor: Jessica Enright}\par\nopagebreak
  	\hspace{\parindent}\textit{Email:~}\texttt{Jessica.Enright@glasgow.ac.uk}\par\nopagebreak
	\vspace{1mm}
	\begin{flushright}
  \textsc{Department of Computing Science\\Sir Alwyn Williams Building\\University of Glasgow\\G12 8QN}
  \end{flushright}
}}

%%%%%%%%%%%%%%%%%%%%%%%%%%%%%%%%%%%%%%%%%%%%%%%%%%

\begin{document}
\maketitle
\renewcommand{\subtitle}[1]{}
\Addresses

\section{Introduction}
\label{sec:intro}
Much work has been done to understand graph problems, such as the Firefighter Problem (which we refer to simply as {\scshape Firefighter}), first proposed by Hartnell \cite{hartnell95} and surveyed in great detail by Finbow and MacGillivray \cite{finbow09}. This problem and other similar ones, such as The Firebreak Problem (again, referred to as {\scshape Firebreak}) have now enjoyed a plethora of results in a purely graph-theoretic context. However, modelling contagion requires relaxing and adapting some assumptions present in the original formulations of these problems. Such assumptions may be that populations are well-mixed (regular graphs) and we have fully predictable outcomes each turn (non-stochastic behaviour). We propose a number of ways to amend these assumptions in this paper to yield a method for contagion modelling that can provide more realistic and contextualised results with agency-specific attributes for each vertex. This may involve providing a defence rating $d\in[0,1]$, where 0 represents a vertex with no protection from the contagion and 1 is a vertex that is fully defended. 

%%%%%%%%%%%%%%%%%%%%%%%%%%%%%%%%%%%%%%%%%%%%%%%%%%

\section{Background}

\subsection{Firefighter}
\label{sec:fire}


The following is a decision formulation given by Finbow and MacGillivray for {\scshape Firefighter} on a tree \cite{finbow09}:

{\scshape Firefighter}\\ \indent
{\scshape Instance:} A rooted graph $(G,r)$ and an integer $k\geq 1$.\\ \indent
{\scshape Question:} Is there a finite sequence $d_1, d_2,\dots d_t$ of vertices of the graph $G$ such that:
	\begin{enumerate}[label=\roman*]
		\item $d_i$ {is neither burned nor defended at time} $i$,
		\item {At time} $t$, no undefended vertex is adjacent to a burning vertex, and
		\item {At least} $k$ {vertices are saved at the end of time} $t$?
	\end{enumerate}


%%%%%%%%%%%%%%%%

\subsection{Agency}
\label{sec:agent}


%%%%%%%%%%%%%%%%%%%%%%%%%%%%%%%%%%%%%%%%%%%%%%%%%%
\newpage

\bibliographystyle{siam}
\bibliography{bibliography.bib}
\nocite{*}

\Addresses

\end{document}  