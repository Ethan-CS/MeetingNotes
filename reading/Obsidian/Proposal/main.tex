\documentclass[hidelinks,a4paper,11pt]{article}
	
\usepackage[title]{appendix}
\usepackage[pdftex]{graphicx}
\usepackage{amsmath, amssymb, amsthm, float, cite,}
\usepackage{url}
\usepackage{multirow}
\usepackage{tikz}
\usepackage{incgraph}
\usepackage{xcolor}
\usepackage{enumitem}
\usepackage{pgfgantt}
\usepackage{pgfplotstable}

\usepackage{url}
\urldef{\technical}\url{http://www.physics.usu.edu/riffe/3750/Lecture%2021.pdf}

\usepackage{datetime} % British format dates

\numberwithin{equation}{section}

\usepackage[cm]{fullpage}

\makeatletter
\def\@supervisor{Dr Jessica Enright}
\newcommand{\supervisor}[1]{
  \def\@supervisor{#1}
}
\def\@studentid{}
\newcommand{\studentid}[1]{
  \def\@studentid{#1}
}
\makeatother

\newcommand{\HRule}{\rule{\linewidth}{0.5mm}}

%%%%%%%%%%%

\begin{document}

\pagenumbering{Roman}
% Title page data

\title{Game-theoretic and probabilistic methods applied to spatial network models of contagion}
\author{Ethan Kelly}
\supervisor{Dr Jessica Enright}
\studentid{2295664K@student.gla.ac.uk}

\begin{titlepage}
\begin{center}

\includegraphics[width=2in]{gulogo_black}~\\[1cm]

\textsc{\LARGE School of Computing Science}\\[1.5cm]
\textsc{\Large Doctoral Program Research proposal}\\[0.5cm]

% Title
\makeatletter
\HRule \\[0.4cm]
{ \huge \bfseries \@title \\[0.4cm] }
\makeatother
\HRule \\[1.5cm]

% Author and supervisor
\noindent
%
%
%
\begin{minipage}{0.4\textwidth}
\begin{flushleft} \large
\emph{Author:}\\
\makeatletter
\Large \@author\\
\texttt{\@studentid}
\makeatother
\end{flushleft}
\end{minipage}%
\begin{minipage}{0.4\textwidth}
%
%
%
\begin{flushright} \large
\emph{Supervisor:} \\
\makeatletter
\Large \@supervisor
\makeatother
\end{flushright}
\end{minipage}
\end{center}
\vspace{1cm}
\tableofcontents
\end{titlepage}

%%%%%%%%%%

\newpage
\pagenumbering{arabic}

\section{Introduction}
\label{sec:intro}

\subsection{Context and foundation of the proposed research}

Networks are an integral tool in modelling many aspects of human life: for instance, with COVID-19 we have experienced the alarming rate at which a biological contagion can spread across networks and the prevalece of `fake news' demonstrates the harmful effects an informational contagion. While these issues are not novel, they show that modelling contagion across networks has remarkably broad uses, which are applicable to much of our experience as social animals.\\

One of the most versatile models used to understand these situations is the Firefighter Problem. This is a formal, discrete-mathematical way of modelling an outbreak (originally a a fire) by Hartnell in 1995. \cite{hartnell95} The fire is passed to any unchanged vertices adjacent to it; that is, if they are not already on fire or `protected.' Each turn, we have the choice of protecting some vertex - this is how we `fight' the fire. The problem centres around the best strategies for playing this game in an outbreak modelled on one of many types of networks, and we have a handful of results already available to us. For instance, we know that the problem is NP-complete for trees of maximum degree 3 \cite{finbow07} but 0-1 integer programming can yield a solution in polynomial time for some subclass of trees. \cite{macgillivray03}. As given by Finbow and MacGillivray, the formalised definition of the deterministic Firefighter Problem is as follows:\footnote{Note that MVS$(G, r; d)$ represents the maximum number of vertices it is possible to save in $G$ where a fire breaks out at a vertex $r$.}

\begin{figure}[ht]
\label{fig:detfire}
\textbf{\\ \large\sc Firefighter}\\
\textbf{\large\sc Instance:~} A rooted graph $(G, r)$ and an integer $k \geq 1$.\\
\textbf{\large\sc Question:~} Is MVS$(G, r; 1) \geq k$? In other words, is there a finite sequence $d_1, d_2, \dots,d_t$ of vertices of $G$ such that if the fire breaks out at $r$, then:\\
~~~I. ~~Vertex $d_i$ is neither burning nor defended at time $i$,\\
~~~II. ~At time $t$ no undefended vertex is adjacent to a burning vertex, and\\
~~~III. At least $k$ vertices are saved at the end of time $t$? \cite{finbow09}
%\caption{Formalisation of the deterministic Firefighter problem }
\end{figure}


\subsection{Proposed improvements to existing work}

The Firefighter Problem is well suited to modelling infectious disease - to demonstrate, consider the following example. First, let vertices represent individuals who can become infected under some determined conditions. The defensive move each turn would be the ability (for instance) to vaccinate $d$ individuals. We aim to find an optimal strategy for performing these protective actions to prevent as much damage as possible. However, there are inherent limitations in deterministic formulations of the Firefighter Problem. In our example, the deterministic approach limits us to binary operations on vertices: they are either protected or not and either on fire or not. The structure is too rigid, lacking authentic or useful representation of agency in vertices modelled as individuals. We propose addressing this in our research with probabilistic methods: we impose that a vertex has a probability $p\in[0,1]$\footnote{This is zero if the vertex is protected and one if there are sufficiently many infected neighbours and the vertex is not protected.} of contracting the disease in the next turn if one of its adjacent vertices (neighbours) has already contracted the disease. Then, the probability of infection increases the more neighbours have contracted the disease around the vertex in question: this is a {\it stochastic} approach to the Firefighter Problem, which Tennenholtz et al. point out is well suited to modelling as a Markov decision process, \cite{tennenholtz17} a system where the outcomes of given situations are decided in part by assigned probabilities and in part by a deterministic decision maker. We conjecture that the above influence-based example can be shown to be NP-hard using the fact that the Firefighter Problem is NP-hard and there (by assumption) exists a polynomial-time reduction from the Firefighter Problem to our more involved example. By this reasoning, this model with any more complex rules applied to consider a wider contextualisation is also likely to be NP-hard.\\

We show in the literature review that currently, game-theoretic and stochastic approaches to the Firefighter Problem tend to be distinct. Most models presented are given to further knowledge in a game theory {\it or} a stochastic context, infrequently both. We claim that the use of game theoretic methods to determine optimal strategies is insufficient for a satisfactory model, as is exclusive use of probabilistic methods. Instead, our approach would give rise to a model which combines these two distinct veins, giving a more nuanced and dynamic model capable of providing results not only in the crucial field of disease, but in other contagions as well (for instance fires, memes and so on). We would use a {\it stochastic} game theoretic approach in modelling human behaviour, thereby distinguishing our work from the limitations of previous single-track approaches.


\section{Literature Review}
\label{sec:litreview}
%In this section, we aim to outline some of the influential work that relates to the proposed work.\\

\subsection{Graph Classes}

Owing to our firm motivation to model contagion more realistically, it is necessary for us to consider a variety of graph classes to find those most suited to our particular uses. We begin with `scale-free' networks: this is a network where the distribution of vertex degrees follows a statistical power law. It has been shown that in a scale-free network, the more biased a policy is towards the better connected hubs, the better the chance of eradicating a virus we have.\footnote{Interestingly, Dezs\H{o} and Barab\'{a}si show that such policies are more cost effective and reduce the quantities of cures or immunisations required to have the epidemic threshold supersede the rate of infection.} \cite{dezso02} We can view the internet as a scale-free network and use this to examine the distribution of content across social networks and between platforms \cite{obrien19} and how this circulation of information can influence individuals in such models. \cite{gleeson14} From a computing perspective, our infrastructure relies increasingly upon the internet, meaning the ability to study the spread of a computer virus outbreak or how media platforms can circulate information would give us valuable insight into what may be ahead of us as our integration with online realities progresses. We propose that our method of modelling would permit the study of this scenario in detail.\\

Another significant class to consider is the `small-world' network: here, most vertices are not neighbours of each other, but are connected to one another by a short route of connections to other vertices. \cite{porter12} In such a network, we can find a simple but effective model that gives us percolation thresholds for the control of disease. \cite{newman02} By using a `small world' model, we can instantiate various constraints on the neighbours a vertex might have when generating, for instance a random graph. This would permit a more representative model of human societal structures, thereby providing a more accurate and relevant outcome.


\subsection{Game-theoretic (deterministic) approaches}

Finbow and MacGillivray provide a literature review that serves as an excellent reference for the deterministic Firefighter Problem.\cite{finbow09} They provide a firm foundation for the problem and detail an array of potential applications along with previous work in such areas. They note that there is a good deal of writing surrounding contagion of diseases using certain graph classes, hence their survey proves a useful roadmap to begin assessing the literature in this area.\\

A notable concept that could be used to find the optimal solutions to the example referred to above is that of the {\it surviving number.} This is the maximum number of vertices that we are capable of protecting when a fire breaks out at some vertex $v$. If we denote this by $S(v)$, the surviving number for a fire beginning at a vertex $v$, we then find the surviving rate $\rho(G)$ as given by Leizhen and Weifan \cite{leizhen09} as $$\displaystyle\rho(G)=\sum_{v\in V}\frac{S(v)}{n^2}.$$That is, the mean percentage of vertices that it is possible to save for a fire breaking out at any vertex $v$ on the graph. They also find that, for any tree of $n$ vertices, the surviving rate is strictly bounded below by $1-\sqrt{2/n}$, for any outerplanar graph this rate is bounded below by ${1}/{6}$ and for any Halin graph with 5 or more vertices, this bound is ${3}/{10}.$\cite{leizhen09} Using this notion of surviving rates, we envisage a clear method of contextualising and evaluating the results of our proposed model.\\

\subsection{Stochastic approaches}

To base our model in a stochastic framework, we will also consider the Random Graph as generated by the Erd\H{o}s-R\'{e}nyi model, specifically in the form presented by Edgar Gilbert. In this model, a fixed probability is provided for each edge to be present or not and this is independent of the probabilities of other edges. \cite{gilbert59} This could be used in our research to generate random graphs with the small-world and other sociological principles to give a more accurate and fruitful model for contagion.\\

A starting example of another stochastic approach is given by Tennenholtz et al. - in a network with a stochastic infection spreading across the vertices, they present a simple defensive strategy of vaccinating neighbours of infected nodes. They show that this approach is optimal on regular trees for a large enough budget. \cite{tennenholtz17} While they present the budget necessary in such an outbreak based on the number of infections, we propose a modelling framework that directly introduces stochastic budget constraints to find how an organisation with fewer funds could best address an outbreak, using game-theoretic methods to find such an optimal strategy.\\

\subsection{Stochastic Game Theory and Human Behaviour}

One of the most prominent criticisms of traditional game theory centres upon the assumptions made regarding agents: if we remove the assumptions of perfect rationality and foresight, opting instead for governing principles such as learning dynamics and introspection, we arrive at `stochastic game theory.' \cite{goeree99} We propose that this would be a far more relevant framework to use in finding strategies than the standard version of game theory. There has been some work done in the field of stochastic game theory and Markov decision processes, the results of which being very encouraging for our purposes: notably, Bowling and Veluso give several examples that could give rise to modelling conditions in our research. \cite{bowling00} One notion of particular interest is that of reinforcement learning: in this form of solution, we are not given - as with traditional game theory - perfect foresight. Rather, any information about the environment must be gained through observation. We aim to give solutions to a model where we consider a single agent trying to avoid contracting a virus and what their optimal strategy would be in various starting conditions. Indeed, we would like to consider the agent's optimal strategy involving learning about the environment as well as avoiding contagion, where environmental learning allows the agent to better avoid contagion (for instance, learning who is infected by testing). \\

Owing to our desire to create a model more reflective of genuine human behaviour, we consider the literature already available in modelling human behaviour and we will use this to inform our graph generation and game strategies. The most pertinent information for our purposes is in the field of human decisions. We are not random: we form decisions based on past experience, our interests and other social and environmental factors. Kennedy provides several basic psychological principles that we aim to instantiate in our modelling behaviour, including ascertaining the motivations of agents, information processing methods and emotional, intuitive and unconscious factors that result in decisions regarding behaviours. \cite{kennedy12} Using this psychological research in a stochastic formulation of the Firefighter Problem, we believe that we would generate a much more representative and fruitful model for the spread of disease through a given network.\\


\section{Aims and Objectives}
\label{sec:aims}

To see how an adapted Firefighter Problem might look, we consider the following: a disease in a graph $G=\langle V,E\rangle$ begins at vertex $v_1$. The function $f:V\rightarrow A$ describes the assignment of an action in $A$ to given vertices. How much of the graph can we save from the spread of disease with $d$ influences per discrete time interval? Influence here has a number of applications: it may mean a government vaccination program, where we influence individuals through campaigns to get vaccinated. Another context may be a social distancing drive, trying to prevent people coming into contact with one another in order to halt contagion. \\

Our research would begin by formalising the above example further, allowing for the introduction of stochastic approaches to vertex behaviour in the particular context we wish to consider. Several methods of optimisation can be pursued in our initial research, beginning with the {\it greedy algorithm:} assign a {\it weight} value $w(v)$ to each vertex $v$, the number of unprotected and uncontaminated vertices directly adjacent to it. The algorithm might then save or influence the vertex with largest weight each turn. In general, however this is not optimal: for instance, on trees the greedy algorithm is a $\frac{1}{2}$-approximation for the firefighter problem. \cite{finbow09} Thus, while this is a good starting point for a strategy, we aim to generate further algorithms that serve as closer approximations where the problem proves NP-hard (such as the Firefighter Problem for trees of maximum degree 3, as we have seen \cite{finbow07}).\\

Another approach that may prove useful in our research is that of integer programming: we would begin by studying and generalising the 0-1 integer programming approach suggested by MacGillivray and Wang \cite{macgillivray03} to our particular, stochastically founded model and seeing whether this could provide a route to optimal strategy. Further, an iterative improvement (heuristic) search is another likely fruitful path to pursue. We envisage that these approaches will be useful in obtaining relatively successful approximations for optimal approaches and that this will be a potential basis of more analytic approaches where possible. These objectives are listed in order of importance in the following section, determining how and when they are to be studied in our context.

\subsection{Objectives}
\label{sec:objectives}

\begin{enumerate}
	\item Create a stochastic framework for the Firefighter Problem that represents human behaviour for vertices in an authentic and useful way.
	\begin{enumerate}
		\item Generate and analyse a general version of the Firefighter Problem with implicit probabilistic properties to represent agents and their behaviours.
		\item Use stochastic game-theoretic methods to determine the optimal strategy for defending the population from the spread of disease. Investigate whether this can be resolved in general.
		\item Use similar strategies with further constraints, such as a limited budget for vaccinations, and determine whether the optimal strategy is still the same as without such constraints.
	\end{enumerate}
	\item Specify scenarios of interest where our model will be applicable.
	\begin{enumerate}
		\item Use the model to examine strategies used to control the spread of COVID-19 by countries across the world, how effective they are in general and how effective they can be with constraints such as budget or communication infrastructure limitations.
		\item Apply the model to the spread of false information and see how this can best be protected against for individual users, taking into account the probabilities that given individuals will adopt such information as belief.
		\end{enumerate}
	\item Assessment of the utility of approximation algorithms (such as the greedy algorithm) in determining an optimal outcome in the version of the Firefighter Problem we have produced.
	\item Assessment of the utility of integer programming approaches to the general and specific cases we generate of the modified Firefighter Problem.
	\item Assessment of an heuristic search approach to determining optimal solutions in the general and specific case of the modified Firefighter Problem.
	\begin{enumerate}
		\item Trade-off: is it theoretically possible to solve our Firefighter Problem analytically?
		\item Investigate algorithms such as local search where such direct analysis is not feasible.
	\end{enumerate}
\end{enumerate}

As mentioned in point 1, the core aim of this research is to produce a realistic and representative version of the Firefighter Problem with sufficient generality to serve a number of purposes, from disease spread to information dissemination. Obtaining the least-worst outcome of an epidemic scenario or minimising the maximum outcome in the case of false information circulation could apply to programs such as the current UK Government SHARE campaign to combat false information on social media. \cite{govt20} To elaborate on point 2, we are particularly interested in examining the measures taken to limit the spread of COVID-19 globally. This would allow us to compare strategies used by different nations and examine the types of networks where they should be effective.\\ %For example, while social distancing appears to have been effective in densely populated areas, other strategies may be more useful in less dense populations where distances between individual places of abode are already much larger on average. To model this, we would examine the results obtained by focusing resources on large, well-connected hubs of vertices, as suggested by Dezs\H{o} and Barab\'{a}si. \cite{dezso02}\\

By fulfilling the objectives outlined here, we would generate a framework for modelling a diverse set of scenarios in a novel and fruitful way. This is owing to the stochastic approach in both the inherent properties of the graph and in the game-theoretic methods we use to find optimal solutions. In doing so, we add to the base of knowledge for protecting populations from disease, optimise computer networks to protect as many machines as possible from viruses and many more network-applicable examples.\\

\section{Timeline}
\label{sec:timeline}
Throughout the four years of the project, we envisage that the most time-consuming stage of research will be developing an initial model. We have, in the Gantt chart below, given opportunity in later sections to re-evaluate and adjust the model according to the findings of other aspects of the proposed research. In doing so, the central portion of the work - a viable and representative model of contagion accounting for human behaviour - will be as well-informed as possible. It is likely that the model will be adjusted and improved at any stage of research, but the times indicated function as important reminders to do so. The other section given a large amount of dedicated time is the final write-up in the last six months of the project: here, we will collate all information gathered and results derived in order to present the most comprehensive and useful thesis from this body of work possible.

\begin{figure}[ht]
  \includegraphics[width=\linewidth]{timeline.png}
  \caption{Gantt Chart showing the timeline for the proposed research, with numbers and letters corresponding to Section \ref{sec:objectives}.}
  \label{fig:timeline}
\end{figure}


\newpage
\pagenumbering{Roman}

\bibliographystyle{siam}
\bibliography{bibliography.bib}
\nocite{*}


\end{document}  